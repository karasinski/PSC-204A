\documentclass[onecolumn,10pt]{jhwhw}

\usepackage{epsfig} %% for loading postscript figures
\usepackage{amsmath}
\usepackage{graphicx}
\usepackage{grffile}
\usepackage{pdfpages}
\usepackage{algpseudocode}
\usepackage{wrapfig}
\usepackage{pgfplots}
\usepackage{amsfonts}
\usepackage{booktabs}
\usepackage{siunitx}
\usepackage{commath}

% Default fixed font does not support bold face
\DeclareFixedFont{\ttb}{T1}{txtt}{bx}{n}{12} % for bold
\DeclareFixedFont{\ttm}{T1}{txtt}{m}{n}{12}  % for normal

% Custom colors
\usepackage{color}
\usepackage{listings}
\usepackage{framed}
\usepackage{caption}
\usepackage{bm}
\captionsetup[lstlisting]{font={small,tt}}

\definecolor{mygreen}{rgb}{0,0.6,0}
\definecolor{mygray}{rgb}{0.5,0.5,0.5}
\definecolor{mymauve}{rgb}{0.58,0,0.82}

\lstset{ %
  backgroundcolor=\color{white},   % choose the background color; you must add \usepackage{color} or \usepackage{xcolor}
  basicstyle=\ttfamily\footnotesize, % the size of the fonts that are used for the code
  breakatwhitespace=false,         % sets if automatic breaks should only happen at whitespace
  % breaklines=true,                 % sets automatic line breaking
  captionpos=b,                    % sets the caption-position to bottom
  commentstyle=\color{mygreen},    % comment style
  deletekeywords={...},            % if you want to delete keywords from the given language
  escapeinside={\%*}{*)},          % if you want to add LaTeX within your code
  extendedchars=true,              % lets you use non-ASCII characters; for 8-bits encodings only, does not work with UTF-8
  frame=single,                    % adds a frame around the code
  keepspaces=true,                 % keeps spaces in text, useful for keeping indentation of code (possibly needs columns=flexible)
  columns=flexible,
  keywordstyle=\color{blue},       % keyword style
  language=Python,                 % the language of the code
  morekeywords={*,...},            % if you want to add more keywords to the set
  numbers=left,                    % where to put the line-numbers; possible values are (none, left, right)
  numbersep=5pt,                   % how far the line-numbers are from the code
  numberstyle=\tiny\color{mygray}, % the style that is used for the line-numbers
  rulecolor=\color{black},         % if not set, the frame-color may be changed on line-breaks within not-black text (e.g. comments (green here))
  showspaces=false,                % show spaces everywhere adding particular underscores; it overrides 'showstringspaces'
  showstringspaces=false,          % underline spaces within strings only
  showtabs=false,                  % show tabs within strings adding particular underscores
  stepnumber=1,                    % the step between two line-numbers. If it's 1, each line will be numbered
  stringstyle=\color{mymauve},     % string literal style
  tabsize=4,                       % sets default tabsize to 2 spaces
}

\author{John Karasinski}
\title{Homework \# 7}

\begin{document}
%\maketitle

\problem{}
In your own words describe how RM ANOVA can be used to test for differences between groups, differences between repeated observations, and differences in observations as a function of group membership. These tests have specific names we discussed in lecture.\\

\begin{enumerate}
\item Groups Hypothesis (Test of Levels)
\begin{enumerate}
\item Do groups score similarly on the collected set of measures?
\item This test checks if the mean of the groups over all measurement times is the same. This does not mean that all the groups performed the same at each time point, just that their average performance is the same over all time points. Mathematically, this is simple measuring the relative contributions of between-group and within-group contributions to the total sum of squared errors. If the group `levels' are significantly different, then the equal levels null hypothesis is rejected.
\end{enumerate}
\item Flatness Hypothesis (Test of Flatness)
\begin{enumerate}
\item Is the dependent variable similar across assessments?
\item This test checks if the difference in the response between assessments (i.e., the slope) is nonzero. The flatness null hypothesis is that the slope between all assessments is zero, and that the results are flat. This is checked for each group, and is effectively a within-subjects test. If the slope is statistically significantly nonzero, then there is a within groups main effect of time.
\end{enumerate}
\item Parallelism Hypothesis (Test of Parallelism)
\begin{enumerate}
\item Do different groups have parallel profiles?
\item This test checks if the slope between each set of assessments is the same across all groups. This is equivalent to a one-way ANOVA on the slopes, we are checking that there is no interaction of the within-subjects factor with the between-subjects factor.
\end{enumerate}
\end{enumerate}

\problem{}
Why do we say that in RM ANOVA time is treated as a categorical variable?

RM ANOVA treats time as a categorical factor. Time is considered a set of discrete, fixed conditions in which each subject contributes one outcome variable per time condition. The ANOVA model for describing mean change over time uses as many parameters as there are discrete occasions. This is also known as a saturated means model---the model for time is saturated by using all possible degrees of freedom for differences across conditions of the time variable. The goal of the ANOVA model is not to predict or summarize the pattern of means, but simply to reproduce the observed mean per occasion using fixed effects equal to the number of occasions minus 1 (given that the fixed intercept is already included).

\problem{}
What is the difference between polynomial contrasts and comparisons of means between different observation periods?

Polynomial contrast tests can be used to test which level of polynomial (linear, quadratic, cubic) explains the relation under study.

\problem{}
Researchers were interested in how confidence in students fluctuates from the end of sophomore year through the end of senior year in college. Further, researchers were interested in whether students attended a private (i.e., Harvard, Stanford, and Yale) or public (i.e., UMass---Boston, San Francisco State University, and Southern Connecticut State University) university would influence end-of-year student confidence ratings, and possible trajectories.

Use the data set “confidence.csv” to answer the following questions. For your information, t1-t3 indicate time of observation 1-3; i.e., end of sophomore year, end of junior year, and end of senior year, respectively. The variable public indicates whether the student attended a public (public = 1) or private (public = 0) university.\\

\noindent1. Convert the wide format data set to a long format data set. Remember that you will need to create an ID variable, and an Observation/Time variable. Show syntax and the header and footer of the long format data set.

\begin{lstlisting}
import pandas as pd

df = pd.read_csv('confidence.csv')

# Convert to long format
r = pd.melt(df.reset_index(),
            value_vars=['t1', 't2', 't3'],
            id_vars=['index', 'public'],
            var_name='Time', value_name='Observation')

# Some basic sorting
r = r.sort(['index', 'Time']).reset_index(drop=True)

# Rename first column to ID
cols = r.columns.tolist()
cols[0] = 'ID'
r.columns = cols

# Print head and tail of new dataframe
print(r.head())
print(r.tail())
\end{lstlisting}
\noindent OUTPUT
\begin{lstlisting}[language={}]
   ID  public Time  Observation
0   0       0   t1            4
1   0       0   t2            6
2   0       0   t3            6
3   1       0   t1            3
4   1       0   t2            5

     ID  public Time  Observation
295  98       1   t2            4
296  98       1   t3            4
297  99       1   t1            4
298  99       1   t2            5
299  99       1   t3            4
\end{lstlisting}

\noindent2. Test whether public and private universities differed in their confidence scores. Conduct any pairwise comparison necessary. Report your conclusions.

\begin{lstlisting}
print(anova_lm(ols("Observation ~ C(public)", df).fit(), typ=2))
\end{lstlisting}
\noindent OUTPUT
\begin{lstlisting}[language={}]
               sum_sq   df          F        PR(>F)
C(public)   48.803333    1  32.419994  2.975196e-08
Residual   448.593333  298        NaN           NaN
\end{lstlisting}

A one-way between subjects ANOVA was conducted to compare the effect of different types of universities on student confidence for students at public and private universities. There was a significant effect of amount of sugar on words remembered at the $p<.001$ level for the two conditions $[F(1, 298) = 32.4, p = 2.98e-08]$. Comparisons of the mean indicated that the mean confidence scores for the private university students $(M=3.77, SD=1.30)$ was significantly different than the public university students $(M=2.96, SD=1.15)$.\\

\noindent3. Test whether confidence differs over time. Do this by comparing all observations to each other. Also do this by testing the maximum allowable number of polynomial contrasts. Report your conclusions for both tests.

\begin{lstlisting}
from patsy import dmatrix

# This is equivalent to R's contr.poly
p = dmatrix("C(df.Time, Poly())", df)
poly = pd.DataFrame(p, columns=['Intercept', 'Linear', 'Quadratic'])
df = pd.concat((df, poly), axis=1)
print(anova_lm(ols("Observation ~ (Time)", data=df).fit()))
print(anova_lm(ols("Observation ~ Linear + Quadratic", data=df).fit(), typ=2))
\end{lstlisting}
\noindent OUTPUT
\begin{lstlisting}[language={}]
           df      sum_sq   mean_sq         F    PR(>F)
C(Time)     2   13.406667  6.703333  4.113494  0.017291
Residual  297  483.990000  1.629596       NaN       NaN

               sum_sq   df         F    PR(>F)
Linear       7.605000    1  4.666801  0.031551
Quadratic    5.801667    1  3.560187  0.060156
Residual   483.990000  297       NaN       NaN
\end{lstlisting}

\noindent4. Test for a possible interaction between variables tested in question 2 (university type) and 3 (mean differences between observations and significant trends in confidence). Conduct any follow-up analyses necessary if there is a significant interactions. Report your conclusions for all tests.

\begin{lstlisting}
print(anova_lm(ols("Observation ~ C(Time)*C(public)", data=df).fit(), typ=2))

# Test for simple effects
print(anova_lm(ols("Observation ~ C(Time)", data=df.query('public == 0')).fit(), typ=2))
print(anova_lm(ols("Observation ~ C(Time)", data=df.query('public == 1')).fit(), typ=2))
print(anova_lm(ols("Observation ~ C(public)", data=df.query('Time == 1')).fit(), typ=2))
print(anova_lm(ols("Observation ~ C(public)", data=df.query('Time == 2')).fit(), typ=2))
print(anova_lm(ols("Observation ~ C(public)", data=df.query('Time == 3')).fit(), typ=2))
\end{lstlisting}
\noindent OUTPUT
\begin{lstlisting}[language={}]
                       sum_sq   df          F        PR(>F)
C(Time)             13.406667    2   4.995134  7.356729e-03
C(public)           48.803333    1  36.366858  4.897455e-09
C(Time):C(public)   40.646667    2  15.144371  5.496327e-07
Residual           394.540000  294        NaN           NaN

              sum_sq   df          F    PR(>F)
C(Time)    37.213333    2  12.685187  0.000008
Residual  215.620000  147        NaN       NaN

          sum_sq   df        F    PR(>F)
C(Time)    16.84    2  6.91784  0.001345
Residual  178.92  147      NaN       NaN

           sum_sq  df         F    PR(>F)
C(public)    0.81   1  1.107113  0.295298
Residual    71.70  98       NaN       NaN

           sum_sq  df         F    PR(>F)
C(public)    4.00   1  2.911468  0.091119
Residual   134.64  98       NaN       NaN

           sum_sq  df          F        PR(>F)
C(public)   84.64   1  44.073964  1.766331e-09
Residual   188.20  98        NaN           NaN
\end{lstlisting}

\begin{table}[h!]
\begin{center}
\begin{tabular}{l r r r r r}
\toprule
Source & SS & df & F & $PR(>F)$ \\
\midrule
Time         &   13.40 &   2 &  5.00 & $<.001$ \\
Public       &   48.80 &   1 & 36.37 & $<.001$ \\
Interaction  &   40.64 &   2 & 15.14 & $<.001$ \\
Residual     &  394.54 & 294 &       &         \\
\bottomrule
\end{tabular}
\end{center}
\caption{Factorial ANOVA Results}
\end{table}


\begin{table}[h!]
\begin{center}
\begin{tabular}{l r r r r r}
\toprule
Source & $SS$ & $df$ & $F$ & $PR(>F)$ \\
\midrule
\it{Public} & & & & \\
\hspace{1em} Private     & 37.21 &  2  & 12.7 & $<.001$ \\
\hspace{1em} Public      & 16.84 &  2  &  6.9 & $ .001$ \\
\it{Time} & & & & \\
\hspace{1em} 1      &  0.81 &  1  &  1.1 & $ .001$\\
\hspace{1em} 2      &  4.00 &  1  &  2.9 & $ .295$\\
\hspace{1em} 3      & 84.64 &  1  & 44.1 & $<.001$\\
\bottomrule
\end{tabular}
\end{center}
\caption{Results of Simple Effects Analysis}
\end{table}

\noindent5. Create a plot of the data that represents the trajectory for public and private students. Make sure the trajectories have some indication of variability around the average trajectory. Sufficiently label and describe your figure.

\end{document}
