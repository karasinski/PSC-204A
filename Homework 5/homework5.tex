\documentclass[onecolumn,10pt]{jhwhw}

\usepackage{epsfig} %% for loading postscript figures
\usepackage{amsmath}
\usepackage{graphicx}
\usepackage{grffile}
\usepackage{pdfpages}
\usepackage{algpseudocode}
\usepackage{wrapfig}
\usepackage{pgfplots}
\usepackage{amsfonts}
\usepackage{booktabs}
\usepackage{siunitx}
\usepackage{commath}

% Default fixed font does not support bold face
\DeclareFixedFont{\ttb}{T1}{txtt}{bx}{n}{12} % for bold
\DeclareFixedFont{\ttm}{T1}{txtt}{m}{n}{12}  % for normal

% Custom colors
\usepackage{color}
\usepackage{listings}
\usepackage{framed}
\usepackage{caption}
\usepackage{bm}
\captionsetup[lstlisting]{font={small,tt}}

\definecolor{mygreen}{rgb}{0,0.6,0}
\definecolor{mygray}{rgb}{0.5,0.5,0.5}
\definecolor{mymauve}{rgb}{0.58,0,0.82}

\lstset{ %
  backgroundcolor=\color{white},   % choose the background color; you must add \usepackage{color} or \usepackage{xcolor}
  basicstyle=\ttfamily\footnotesize, % the size of the fonts that are used for the code
  breakatwhitespace=false,         % sets if automatic breaks should only happen at whitespace
  % breaklines=true,                 % sets automatic line breaking
  captionpos=b,                    % sets the caption-position to bottom
  commentstyle=\color{mygreen},    % comment style
  deletekeywords={...},            % if you want to delete keywords from the given language
  escapeinside={\%*}{*)},          % if you want to add LaTeX within your code
  extendedchars=true,              % lets you use non-ASCII characters; for 8-bits encodings only, does not work with UTF-8
  frame=single,                    % adds a frame around the code
  keepspaces=true,                 % keeps spaces in text, useful for keeping indentation of code (possibly needs columns=flexible)
  columns=flexible,
  keywordstyle=\color{blue},       % keyword style
  language=Python,                 % the language of the code
  morekeywords={*,...},            % if you want to add more keywords to the set
  numbers=left,                    % where to put the line-numbers; possible values are (none, left, right)
  numbersep=5pt,                   % how far the line-numbers are from the code
  numberstyle=\tiny\color{mygray}, % the style that is used for the line-numbers
  rulecolor=\color{black},         % if not set, the frame-color may be changed on line-breaks within not-black text (e.g. comments (green here))
  showspaces=false,                % show spaces everywhere adding particular underscores; it overrides 'showstringspaces'
  showstringspaces=false,          % underline spaces within strings only
  showtabs=false,                  % show tabs within strings adding particular underscores
  stepnumber=1,                    % the step between two line-numbers. If it's 1, each line will be numbered
  stringstyle=\color{mymauve},     % string literal style
  tabsize=4,                       % sets default tabsize to 2 spaces
}

\pgfmathdeclarefunction{gauss}{2}{%
  \pgfmathparse{1/(#2*sqrt(2*pi))*exp(-((x-#1)^2)/(2*#2^2))}%
}

\author{John Karasinski}
\title{Homework \# 5}

\begin{document}
%\maketitle

\problem{}
Calculate the between and within df for the following conditions.

\part Five conditions (e.g., placebo, dose 1, dose 2, dose 3, and dose 4) with N = 100?
\begin{equation*}
\begin{split}
df_{\mbox{between}}&=4\\
df_{\mbox{within}}&=N-df_{\mbox{between}}-1\\
                  &=100-4-1=95
\end{split}
\end{equation*}

\part Three conditions with N = 45?
\begin{equation*}
\begin{split}
df_{\mbox{between}}&=2\\
df_{\mbox{within}}&=N-df_{\mbox{between}}-1\\
                  &=45-2-1=42
\end{split}
\end{equation*}

\part Eight conditions with N = 200?
\begin{equation*}
\begin{split}
df_{\mbox{between}}&=7\\
df_{\mbox{within}}&=N-df_{\mbox{between}}-1\\
                  &=200-7-1=192
\end{split}
\end{equation*}

\part Four conditions with N = 36?
\begin{equation*}
\begin{split}
df_{\mbox{between}}&=3\\
df_{\mbox{within}}&=N-df_{\mbox{between}}-1\\
                  &=36-3-1=32
\end{split}
\end{equation*}
\problem{}
Use the F-table provided in the hw05 folder on smartsite.

\part What is the critical F-value given an alpha of .05 and df between of 2 and df within of 20?
$$3.49$$

\part What is the critical F-value given an alpha of .05 and df between of 4 and df within of 30?
$$2.69$$

\part What is the critical F-value given an alpha of .05 and df between of 3 and df within of 100?
$$2.70$$

\part What is the critical F-value given an alpha of .01 and df between of 5 and df within of 50?
$$3.41$$

\problem{}
In your own words, explain what the between and within SS from an ANOVA table represents? Why do you think these values are divided by their respective df in order to estimate MS and then the F value? Why does the one-way ANOVA not help researchers understand where differences between conditions are, but can be used to determine if there are significant differences between conditions?

The $SS_{between}$ and $SS_{within}$ from the ANOVA tables represent the amount of variation between groups and the amount of variation between all subjects. These values are divided by their respective df in order to normalize the values against one another. If we did not do this renormalization it would give undo weight to one or the other $SS$ values. The F value from the ANOVA table (and ultimately the $p$ value), tell the researcher that there is a significant devation away from $H_0$ for the groups and the subjects. As $H_0$ is the assumption that all the groups have the same variation, the opposite is then true---that all the groups do not have the variation. This statement does not indicate which groups are different, and nothing about one-way ANOVA selects any particular group for comparison. Due to this, post-hoc tests are used to look at the differences between selected groups.

\problem{}
Complete by hand, you may use a calculator, show all work. For intermediary calculations (e.g., mean and sd) round to four decimal places. Round your final answer to two decimal places.

\begin{center}
\begin{tabular}{l r r r r r}
\toprule
Source & df & SS & MS & F & P \\
\midrule
Between &    & 2510.5 & & & \\
Within  & 12 &        & & & \\
Total   & 14 & 2671.7 & & & \\
\bottomrule
\end{tabular}
\end{center}

\begin{align*}
df_{total} &= df_{between} + df_{within} && \implies df_{between} = 14-12 = 2\\
SS_{total} &= SS_{between} + SS_{within} && \implies SS_{within} = 2671.7-2510.5=161.2\\
MS_{between} &= \frac{SS_{between}}{df_{between}} && \implies MS_{between} = \frac{2510.5}{2} = 1255.25 \\
MS_{within}  &= \frac{SS_{within}}{df_{within}}   && \implies MS_{within}  = \frac{161.2}{12} = 13.4333 \\
F &= \frac{\mbox{MS}_{between}}{\mbox{MS}_{within}} && \implies F = \frac{1255.25}{13.4333} = 93.4432
\end{align*}
Using the F-table provided for this assignment, with $df_{numerator} = 2$ and $df_{denominator} = 12$, we see that $F = 93.4432 > 12.97$, suggesting a $p < .001$. The table can then be completed as:

\begin{center}
\begin{tabular}{l r r r r r}
\toprule
Source & df & SS & MS & F & P \\
\midrule
Between &  2 & 2510.50 & 1255.25 & 93.44 & $<.001$ \\
Within  & 12 &  161.20 & 13.43 & & \\
Total   & 14 & 2671.70 & & & \\
\bottomrule
\end{tabular}
\end{center}

\problem{}
\begin{center}
\begin{tabular}{l r r r r r}
\toprule
Source & df & SS & MS & F & P \\
\midrule
Between &    & 16 &     & & \\
Within  & 95 &    & 1.2 & & \\
Total   & 99 &    &     & & \\
\bottomrule
\end{tabular}
\end{center}

\begin{align*}
df_{total} &= df_{between} + df_{within} && \implies df_{between} = 99-95 = 4\\
SS_{within} &= MS_{within} * df_{within} && \implies SS_{within} = 1.2 * 95 = 114\\
SS_{total} &= SS_{between} + SS_{within} && \implies SS_{total} = 16 + 114 = 130\\
MS_{between} &= \frac{SS_{between}}{df_{between}} && \implies MS_{between} = \frac{16}{4} = 4 \\
F &= \frac{\mbox{MS}_{between}}{\mbox{MS}_{within}} && \implies F = \frac{4}{1.2} = 3.3333
\end{align*}
Using the F-table provided for this assignment, with $df_{numerator} = 4$ and $df_{denominator} = 100 (\approx 95)$, we see that $F = 3.3333 > 2.92$, suggesting $.010 < p < .025$. The table can then be completed as:

\begin{center}
\begin{tabular}{l r r r r r}
\toprule
Source & df & SS & MS & F & P \\
\midrule
Between &  4 &  16 & 4.0 & 3.33 & $.010 < p < .025$ \\
Within  & 95 & 114 & 1.2 & & \\
Total   & 99 & 130 &     & & \\
\bottomrule
\end{tabular}
\end{center}

\end{document}
